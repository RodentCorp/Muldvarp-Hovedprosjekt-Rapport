\documentclass[../main.tex]{subfiles} 
\begin{document}

\section{DR{\O}FTING}

\bigskip

{\itshape\color{blue}
[Vurdering av metode og oppn{\aa}dde resultater. Begrensninger, endringer eller avvik i prosjekt i forhold til plan /
opprinnelig problemstilling - mulige feilkilder. Resultatenes betydning. Dette er en subjektiv vurdering.}

{\itshape\color{blue}
\textbf{\textup{Kommentar}}: Her kan man for eksempel gj{\o}re seg tanker rundt kvaliteten av det arbeidet som er
nedlagt. Er de kildene dere bruker p{\aa}litelige, er det sprik mellom forskjellige kilder (og i s{\aa} fall hvorfor),
er det andre forhold som kan v{\ae}re med {\aa} gj{\o}re noen av de vurderinger og valg dere har gjort usikre? I mindre
oppgaver som denne, kan det ogs{\aa} v{\ae}re en mulighet {\aa} sl{\aa} sammen dette avsnittet med det over.}

{\itshape\color{blue}
Husk at et hovedprosjekt har to resultatm{\aa}l: Det ene er {\aa} komme frem til en l{\o}sning p{\aa} en
problemstilling, det andre m{\aa}let er {\aa} oppn{\aa} en l{\ae}ring om gjennomf{\o}ring av prosjekter. Dette m{\aa}
komme frem i dr{\o}ftingen. Det kan derfor v{\ae}re fornuftig {\aa} dele dette avsnittet i to, der du i den ene delen
gjennomf{\o}rer en dr{\o}fting av det tekniske oppn{\aa}dde resultatet (systemet du har utviklet for eksempel), og der
du i den andre delen gjennomf{\o}rer en dr{\o}fting av selve prosjektet; fungerte prosjektet slik dere valgte {\aa}
organisere det ? Hvilke erfaringer gjorde dere med valgt utviklingsmetodikk (iterrativ/vannfalls/XP etc.). Hvordan
h{\aa}ndterte dere ting som oppstod underveis i prosjektet (aktiviteter som ble st{\o}rre og lengre i tid enn f{\o}rst
planlagt) osv.]}

\end{document}