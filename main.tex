\documentclass[a4paper]{article}
\usepackage[utf8]{inputenc}
\usepackage[T1]{fontenc}
\usepackage{listings}
\usepackage[ngerman,norsk,english]{babel}
\usepackage{amsmath}
\usepackage{amssymb,amsfonts,textcomp}
\usepackage{color}
\usepackage{array}
\usepackage{supertabular}
\usepackage{hhline}
\usepackage{hyperref}
\hypersetup{pdftex, colorlinks=true, linkcolor=blue, citecolor=blue, filecolor=blue, urlcolor=blue, pdftitle=, pdfauthor=, pdfsubject=, pdfkeywords=}
\usepackage[pdftex]{graphicx}
\usepackage{endnotes}
\usepackage[figurename=Figur, tablename=Tabell]{caption}
\usepackage{wrapfig}
\usepackage{lastpage}

\usepackage{fancyhdr}
\usepackage{subfiles}
\usepackage{subcaption}


\definecolor{mygreen}{rgb}{0,0.6,0}
\definecolor{mygray}{rgb}{0.5,0.5,0.5}
\definecolor{mymauve}{rgb}{0.58,0,0.82}

\lstset{ %
  backgroundcolor=\color{white},   % choose the background color; you must add \usepackage{color} or \usepackage{xcolor}
  basicstyle=\footnotesize,        % the size of the fonts that are used for the code
  breakatwhitespace=false,         % sets if automatic breaks should only happen at whitespace
  breaklines=true,                 % sets automatic line breaking
  captionpos=b,                    % sets the caption-position to bottom
  commentstyle=\color{mygreen},    % comment style
  deletekeywords={...},            % if you want to delete keywords from the given language
  escapeinside={\%*}{*)},          % if you want to add LaTeX within your code
  extendedchars=true,              % lets you use non-ASCII characters; for 8-bits encodings only, does not work with UTF-8
  frame=single,                    % adds a frame around the code
  keepspaces=true,                 % keeps spaces in text, useful for keeping indentation of code (possibly needs columns=flexible)
  keywordstyle=\color{blue},       % keyword style
  language=Octave,                 % the language of the code
  morekeywords={*,...},            % if you want to add more keywords to the set
  numbers=left,                    % where to put the line-numbers; possible values are (none, left, right)
  numbersep=5pt,                   % how far the line-numbers are from the code
  numberstyle=\tiny\color{mygray}, % the style that is used for the line-numbers
  rulecolor=\color{black},         % if not set, the frame-color may be changed on line-breaks within not-black text (e.g. comments (green here))
  showspaces=false,                % show spaces everywhere adding particular underscores; it overrides 'showstringspaces'
  showstringspaces=false,          % underline spaces within strings only
  showtabs=false,                  % show tabs within strings adding particular underscores
  stepnumber=2,                    % the step between two line-numbers. If it's 1, each line will be numbered
  stringstyle=\color{mymauve},     % string literal style
  tabsize=2,                       % sets default tabsize to 2 spaces
  title=\lstname                   % show the filename of files included with \lstinputlisting; also try caption instead of title
}

\renewcommand\notesname{}
\renewcommand{\theendnote}{[ \arabic{endnote} ]}
\graphicspath{{bilder/}{../bilder/}}

% Outline numbering
\setcounter{secnumdepth}{2}
\makeatletter
\newcommand\arraybslash{\let\\\@arraycr}
\makeatother
% List styles
\newcommand\liststyleWWviiiNumxiii{%
\renewcommand\theenumi{\arabic{enumi}}
\renewcommand\theenumii{\alph{enumii}}
\renewcommand\theenumiii{\roman{enumiii}}
\renewcommand\theenumiv{\arabic{enumiv}}
\renewcommand\labelenumi{[\theenumi]}
\renewcommand\labelenumii{\theenumii.}
\renewcommand\labelenumiii{\theenumiii.}
\renewcommand\labelenumiv{\theenumiv.}
}
% Page layout (geometry)
\setlength\voffset{-1in}
\setlength\hoffset{-1in}
\setlength\topmargin{1.251cm}
\setlength\oddsidemargin{2cm}
\setlength\textheight{20.242cm}
\setlength\textwidth{16.999cm}
\setlength\footskip{3.302cm}
\setlength\headheight{1.251cm}
\setlength\headsep{1.152cm}
% Footnote rule
\setlength{\skip\footins}{0.119cm}
\renewcommand\footnoterule{\vspace*{-0.018cm}\setlength\leftskip{0pt}\setlength\rightskip{0pt plus 1fil}\noindent\textcolor{black}{\rule{0.25\columnwidth}{0.018cm}}\vspace*{0.101cm}}


% Pages styles
\makeatletter

\fancypagestyle{normal}{%
 	\fancyhead{} % clear all header fields
	\fancyfoot{}
	\fancyhead[LO,LE]{\scshape Høgskolen i Ålesund\\Hovedprosjekt}
 	\fancyhead[RO,RE]{\scshape Side \thepage}
	\renewcommand{\headrulewidth}{0.0pt}
}
\fancypagestyle{frontpage}{%
  	\fancyhead{}
	\fancyfoot{}
	\renewcommand{\footrulewidth}{0.4pt}
	\fancyhead[L]{\scshape \huge Hovedprosjekt}
	\fancyhead[R]{\includegraphics[width=5.9944cm,height=2.0574cm]{hials.png}}
	\fancyfoot[L]{
\parbox{35mm}{\small\ \\ \textbf{Postadresse}\\ Høgskolen i {\AA}lesund \\ N-6025 {\AA}lesund \\Norway}
\parbox{35mm}{\small\textbf{Besøksadresse}\\ Larsg{\aa}rdsvegen 2 \\ \textbf{Internett} \\ www.hials.no}
\parbox{35mm}{\small\textbf{Telefon}\\ 70 16 12 00 \\ \textbf{Epostadresse} \\ postmottak@hials.no}
\parbox{30mm}{\small\textbf{Telefax}\\ 70 16 13 00 \\ \\}
\parbox{30mm}{\small\textbf{Bankkonto}\\ 7694 05 00636 \\ \textbf{Foretaksregisteret} \\ NO 971 572 140}
}

}

\makeatother
\setlength\tabcolsep{1mm}
\renewcommand\arraystretch{1.7}

% Framside variabler
\newcommand{\Tittel}{D1304: Mobil-app til støtte av kurs og undervisning}
\newcommand{\Kandidatnummer}{Kristoffer Strøm Bergset (1709), Terje Nilsson Wallem, Johan Alexander de Lima Hessen}
\newcommand{\Dato}{31.05.13}
\newcommand{\Fagkode}{ID303006}
\newcommand{\Fagnavn}{Hovedprosjekt - Data og automasjon}
\newcommand{\Doktilgang}{}
\newcommand{\Studium}{Bachelor i ingeniørfag - datateknikk}
\newcommand{\Sider}{\pageref{LastPage} / 3}
\newcommand{\Biblnr}{}
\newcommand{\Veiledere}{Mikael Tollefsen}


\begin{document}
\clearpage\setcounter{page}{1}\pagestyle{normal}
\thispagestyle{frontpage}

\begin{flushleft}
\tablefirsthead{}
\tablehead{}
\tabletail{}
\tablelasttail{}
\begin{supertabular}[10cm]{|m{17.077cm}|}
\hline
{\scshape Tittel:}\\
\Large \Tittel
\\\hline
\end{supertabular}
\end{flushleft}

\begin{flushleft}
\tablefirsthead{\hline
\multicolumn{5}{|m{17cm}|}{{\scshape Kandidatnummer(e):}}\\
\multicolumn{5}{|m{17cm}|}{\Kandidatnummer
}\\}
\tablehead{\hline
\multicolumn{5}{|m{17cm}|}{{\scshape Kandidatnummer(e):}}\\
\multicolumn{5}{|m{17cm}|}{~
}\\}
\tabletail{}
\tablelasttail{}
\begin{supertabular}{|m{2.7159998cm}m{3.007cm}m{3.054cm}m{3.55cm}|m{3.90cm}|}
\hline
\multicolumn{1}{|m{2.7159998cm}|}{{\scshape Dato: }} &
\multicolumn{1}{m{3.007cm}|}{{\scshape Fagkode:}} &
\multicolumn{2}{m{6.804cm}|}{{\scshape Fagnavn:}} &
{\scshape Dokument tilgang:}\\
\multicolumn{1}{|m{2.7159998cm}|}{\Dato
} &
\multicolumn{1}{m{3.007cm}|}{\Fagkode
} &
\multicolumn{2}{m{6.804cm}|}{\Fagnavn
} &
\Doktilgang
\\\hline
\multicolumn{3}{|m{9.177cm}|}{{\scshape Studium:}} &
{\scshape Ant sider/Vedlegg:} &
{\scshape Bibl. nr:}\\
\multicolumn{3}{|m{9.177cm}|}{\Studium
} & \centering{\Sider} 
& \Biblnr
\\\hline
\end{supertabular}
\end{flushleft}

\begin{flushleft}
\tablefirsthead{}
\tablehead{}
\tabletail{}
\tablelasttail{}
\begin{supertabular}{|m{17.058999cm}|}
\hline
{\scshape Veileder(e):}\\
\Veiledere
\\\hline
\end{supertabular}
\end{flushleft}

\begin{flushleft}
\tablefirsthead{}
\tablehead{}
\tabletail{}
\tablelasttail{}
\begin{supertabular}{|m{17.058999cm}|}
\hline
{\scshape Sammendrag:}\\
\subfile{"0.1 Sammendrag/sammendrag.tex"}\\
\hline
\end{supertabular}
\end{flushleft}

{\centering\itshape
Denne oppgaven er en eksamensbesvarelse utført av studenter ved Høgskolen i Ålesund.
\par}

\newpage

\subfile{"0 Forord/forord.tex"}

\newpage

\clearpage{\centering\bfseries
INNHOLD
\par}

\setcounter{tocdepth}{3}
\renewcommand\contentsname{}
\tableofcontents

\clearpage

\renewcommand{\arraystretch}{1.1}

\section*{SAMMENDRAG}

\subfile{"0.1 Sammendrag/sammendrag.tex"}

\newpage

\subfile{"0.2 Terminologi/terminologi.tex"}

\subfile{"1 Innledning/innledning.tex"}

\subfile{"2 Teoretisk grunnlag/teoretisk_grunnlag.tex"}

\subfile{"3 Materialer og Metode/metode.tex"}

\subfile{"4 Resultater/resultater.tex"}

\subfile{"5 Drofting/drofting.tex"}

\subfile{"6 Konklusjon/konklusjon.tex"}

\subfile{"7 Referanser/referanser.tex"}

\subfile{"A Vedlegg/vedlegg.tex"}

\end{document}
