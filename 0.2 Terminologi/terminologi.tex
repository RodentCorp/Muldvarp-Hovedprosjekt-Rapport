\documentclass[../main.tex]{subfiles} 
\begin{document}

\chapter*{TERMINOLOGI}

\section*{Begreper}

\textbf{\textit{Aktiviteter}} og \textbf{\textit{Fragmenter}} - I Android er hvert skjermbilde bygd opp av en aktivitet og kan inneholde flere fragmenter. Aktiviteten styrer fragmentene og kan hente fram, vise disse og skjule dem igjen. Et fragment kan være ett helt skjermbilde eller bare en del av ett, så et skjermbilde kan være oppbygd av flere fragmenter. \endnote{\url{http://developer.android.com/guide/components/fragments.html}} I dette prosjektet brukes begrepet “fragment” også til å beskrive moduler med informasjon i webapplikasjonen og mobilapplikasjonen. \newline
\newline
\textbf{\textit{Bug}} - Dette er en feil i kildekoden som resulterer i et uventet og ofte uønsket resultat.\endnote{\url{http://www.techopedia.com/definition/24864/software-bug} - 29.05.2013}\newline
\newline
\textbf{\textit{Chen notasjon}} - Et sett med teknikker utviklet av Peter Chen for å lage entity-relationship modeller (ER modeller). \endnote{\url{http://citeseerx.ist.psu.edu/viewdoc/summary?doi=10.1.1.123.1085} - 19.05.13}\newline
\newline
\textbf{\textit{Commit}} - Dette begrepet refererer i sammenheng med denne rapporten til å bidra med kildekode til revisjonskontrollsystemet Git(se begrepet “GitHub”). \endnote{\url{http://gitref.org/basic/} - 28.05.2013}\newline
\newline
\textbf{\textit{E-læringsapplikasjon}} - Dette begrepet beskriver hele systemet som denne rapporten omhandler\newline
\newline
\textbf{\textit{GET}} - En metode for å hente data over HTTP protokollen. \endnote{\url{http://tools.ietf.org/html/rfc2616}} \endnote{Mike Jasnowski, Java, XML, and Web Services Bible (Hungry Minds Inc, 2002)}  \newline
\newline
\textbf{\textit{GitHub}} - en gratis hostingtjeneste for prosjekter som bruker revisjonskontrollsystemet Git. Git brukes for å dele prosjektfiler mellom utviklere under produksjonen. For mer informasjon om Git se teoretisk grunnlag.\newline
\newline
\textbf{\textit{GUI}} - Graphical User Interface, eller grafisk brukergrensesnitt er en betegnelse på en grafisk representasjon av data på en skjerm\newline
\newline
\textbf{\textit{Mobilapplikasjon}} - Dette begrepet beskriver android applikasjonen som denne rapporten omhandler.\newline
\newline
\textbf{\textit{VMWare vSphere cluster}} - Et virtualiseringsmiljø som kan inneholde mange virtuelle maskiner med forskjellige operativsystemer. \endnote{\url{http://www.vmware.com/products/datacenter-virtualization/vsphere/overview.html}} \endnote{Naveed Yaqub, Comparison of Virtualization Performance: VMWare and KVM(Universitetet i Oslo, 2012), 23}\newline
\newline
\textbf{\textit{Webapplikasjon}} - Dette begrepet beskriver serversiden av systemet som denne rapporten omhandler\newline
\newline
\textbf{\textit{Serialisering}} - Konvertering av data fra objekter i minne til en bitstrøm som kan lagres i en fil eller sendes over et nettverk. \endnote{Else Lervik og Vegard B. Havdal, Programmering i Java 4. utgave (Gyldendal Akademisk, 2009)} \newline
\newline
\textbf{\textit{Sømløst grensesnitt}} - Dette beskriver et grensesnitt hvor to eller flere tjenester kommer sammen i et grensesnitt uten at det oppleves som man bruker forskjellige tjenester.\endnote{\url{http://whatis.techtarget.com/definition/seamless-interface} - 28.05.2013} \newline
\newline
\textbf{\textit{Web scraping}} - Dette er en teknikk for innhenting av data fra en nettside. Man henter ut dataene man vil ha ved å prosessere HTML-koden som vises på en nettside direkte. Med denne teknikken kan man hente ut data fra nesten hvilken som helst nettside. Ulempene med metoden er at den er treg å implementere, og er veldig sårbar for visuelle forandringer hos kildenettsiden. \endnote{\url{http://en.wikipedia.org/wiki/Web_scraping}} \newline
\newline
\textbf{\textit{Web crawler}} - En web crawler er et program som systematisk går igjennom alle nettsider på Internett. Denne kan ha mange hensikter, men vanligvis er hensikten å indeksere sidene og er typisk for store søkemotorer som Google, Bing og Yahoo. \endnote{Tarjei Romtveit, Load-balancing by applying a bayesian learning automata (BLA) scheme in a non-stationary web-crawler network(Universitetet i Agder, 2010), 14} \newline
\newline
\section*{Forkortelser}
\begin{tabular}{ p{2cm} l  }
AJAX	 & Asynchronous JavaScript and XML\\
APK & Android application package file\\
CSS & Cascading Style Sheets\\
EJB & Enterprise JavaBeans\\
ER & Entity Relationship\\
GUI & Graphical User Interface\\
HTML & HyperText Markup Language\\
Java EE & Java Enterprise Edition\\
JAX-RS & Java API for RESTful Web Services\\
JSF & JavaServer Faces\\
JSON & JavaScript Object Relation\\
JSP & JavaServer Pages\\
JVM & Java Virtual Machine\\
NTNU & Norges Teknisk-Naturvitenskapelige Universitet\\
POJO & Plain Old Java Object\\
Regex & Regular Expression\\
REST & Representational State Transfer\\
SDK & Software Development Kit\\
SRU & Search/Retrieval via URL\\
URI & Uniform resource identifier\\
URL & Uniform resource locator\\
WAR & Web application ARchive\\
\end{tabular}


\newpage

\end{document}