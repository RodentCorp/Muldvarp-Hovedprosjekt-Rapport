\documentclass[../main.tex]{subfiles} 
\begin{document}
Dette prosjektet omhandler videreutviklingen av en e-læringsapplikasjon utviklet for Høgskolen i Ålesund i et tidligere prosjekt. Hensikten med e-læringsapplikasjonen er å være et støtteverktøy i undervisningen for både studenter og forelesere.\newline
E-læringsapplikasjonssystemet består av en mobilapplikasjon for Android enheter og en webapplikasjon laget for Java Enterprise Edition plattformen.\newline
\newline
Målet med dette prosjektet er å forbedre applikasjonen ved å gjøre den mer stabil og fleksibel. I tillegg ønsker vi å integrere utvalgte funksjoner fra Fronter, BibSys og TimeEdit i applikasjonen.\newline
\newline
Resultatet ble en stabil e-læringsapplikasjon hvor man kan legge til og fjerne innhold og funksjoner på en fleksibel måte. Den kan også søke etter bøker i skolens bibliotek, vise timeplanen for en gitt klasse eller fag, og fra Fronter kan den presentere meldinger fra lærere, vise tilgjengelige dokumenter og status på innleveringer.\newline
\newline
Denne rapporten inneholder en beskrivelse av hvilke teknologier som ble brukt, prosjekts fremgangsmåte, resultatet av prosjektet, brukerdokumentasjon for mobilapplikasjonen og webapplikasjonen, drøftingen av resultatet og konkluderer med at målene ble oppfylt.
\end{document}