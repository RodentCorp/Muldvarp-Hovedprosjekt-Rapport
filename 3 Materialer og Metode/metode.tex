\documentclass[../main.tex]{subfiles} 
\begin{document}

\section{MATERIALER OG METODE}

\bigskip

{\itshape\color{blue}
[Denne delen skal redegj{\o}re for hvordan man planla {\aa} g{\aa} fram / har g{\aa}tt fram for {\aa} l{\o}se oppgaven
og sannsynliggj{\o}re framgangsm{\aa}ten. Hovedm{\aa}lsettingen med dette avsnittet er {\aa} beskrive alle sider ved
prosjektet som er n{\o}dvendig for {\aa} kunne gjennskape prosjektet/fors{\o}ket ved en senere anledning.}

{\itshape\color{blue}
Dette er et 2-delt avsnitt. P{\aa} den ene siden skal man her beskrive hvordan man har organisert prosjektet, hvordan
man har planlagt prosjektet, hvilke prosedyrer man har etablert for godkjenning av kode og eventuelle delleveranser,
hvordan man har kvalitetssikret arbeidet. I tillegg skal man her beskrive hvordan oppdragsgiver har v{\ae}rt involvert
i prosjektet og avtalte kriterier for fullf{\o}rt oppgave.}

{\itshape\color{blue}
Det skal ogs{\aa} her fremkomme hva som er forventet at tas frem av dokumentasjon i prosjektet (brukerveiledning,
systemdokumentasjon, testrapporter, akseptansetester etc.).}

{\itshape\color{blue}
P{\aa} den andre siden, skal man ogs{\aa} beskrive de tekniske sidene ved prosjektet: }

{\itshape\color{blue}
\textbf{For SW-prosjekter} betyr det hvilken utviklingsmetodikk som er valgt (som da b{\o}r v{\ae}re beskrevet i
kapittel 4), hvilke programmeringsspr{\aa}k er benyttet, hvilke verkt{\o}y og eksterne biblioteker har man benyttet,
hvordan er utviklingsmilj{\o}et satt opp (katalogstruktur, bygg-filer som makefile el.l., avhengigheter mellom
SW-moduler etc.). Dersom det er satt opp et testoppsett for {\aa} teste systemet, slik det vil bli satt opp/installert
hos kunde, s{\aa} skal dette oppsettet beskrives.}

{\itshape\color{blue}
\textbf{For HW/automasjons-prosjekter }skal det her gis en detaljert oversikt over utstyr som er benyttet (multimeter,
skop etc.), m{\aa}lemetoder som er benyttet, samt en detaljert beskrivelse av eventuell
oppstilling/laboppsett/m{\aa}leoppstilling. Det skal ogs{\aa} gis en detaljert beskrivelse av datagrunnlaget som er
benyttet. Husk {\aa} oppgi b{\aa}de kilde og n{\o}yaktighet (kvalitet) til datagrunnlaget. Det kan derfor v{\ae}re
hensiktsmessig {\aa} benytte f{\o}lgende underkapittel ifm. slike prosjekter:]}


\bigskip

\subsection{Data}

\bigskip

{\itshape\color{blue}
[Her beskrives hvilke data som er benyttet i rapporten. Hvilke data som er benyttet, hvor dataene kommer fra (kilde) og
hva som er usikkerheten i dataene.]}


\bigskip

\subsection[Metode]{Metode}

\bigskip

{\itshape\color{blue}
[Her beskriver hvilke metoder som er benyttet til {\aa} bearbeide dataene. Det gjelder f.eks metoder for filtrering,
signalanalyse, m{\o}nstergjenkjenning osv. Alle formel skal v{\ae}re nummerert og alle symboler i metodene skal
v{\ae}re definert under notasjonslisten i terminologikapittelet. }

{\itshape\color{blue}
Der a er{\dots}{\dots}.}

{\itshape\color{blue}
Senere i rapporten refererer du til nummeret p{\aa} metoden.]}


\bigskip

\subsection{Materialer}

\bigskip

{\itshape\color{blue}
[Her beskriver du det praktiske utstyr som er benyttet for {\aa} f{\aa} fram resultatene i rapporten. Det gis her en
detaljert oversikt over utstyr som er benyttet (multimeter, skop etc.), m{\aa}lemetoder som er benyttet, samt en
detaljert beskrivelse av eventuell oppstilling/laboppsett/m{\aa}leoppstilling. Dersom utstyret er en del av produktet
som skal utvikles under prosjektet, beskriver du dette under kapittelet Resultater.]}


\bigskip

\end{document}